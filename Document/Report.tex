\documentclass[conference]{IEEEtran}
\IEEEoverridecommandlockouts
% The preceding line is only needed to identify funding in the first footnote. If that is unneeded, please comment it out.
\usepackage{cite}
\usepackage[utf8]{inputenc}
\usepackage{amsmath,amssymb,amsfonts}
\usepackage{algorithmic}
\usepackage{graphicx}
\usepackage{textcomp}
\usepackage{xcolor}
\def\BibTeX{{\rm B\kern-.05em{\sc i\kern-.025em b}\kern-.08em
    T\kern-.1667em\lower.7ex\hbox{E}\kern-.125emX}}
\begin{document}

\title{Pedestrians detections by adaptative  background mixtured model and histogram of oriented gradients\\
}

\author{\IEEEauthorblockN{Otho Teixeira Komatsu}
\IEEEauthorblockA{\textit{Department of Computer Science} \\ 
\textit{University of Brasília}\\
Brasília, Brasil \\
otho.tk@hotmail.com}
\and
\IEEEauthorblockN{Giordano Süffert Monteiro}
\IEEEauthorblockA{\textit{Department of Computer Science} \\
\textit{University of Brasília}\\
Brasília, Brasil \\
email address}
}

\maketitle

\begin{abstract}
The need of a technology based on pedestrian detection and models to describe a scene from videos has been largerly a research topic, bringing out a diversity of techniques and tools to improve the process. In this report, the algorithm was based on a improved adaptative  background mixtured model, a technique that allows the program to detect distinguish between the moving objects and the background from the scene. To detection of human, histogram of oriented gragients was implemented along with the process of nonmax supression, classifying and tracking through the frame using a  pre-trained Support Vector Machine.
\end{abstract}

\begin{IEEEkeywords}
Adaptative  background mixtured model, Histogram of oriented gragients,Support Vector Machine .
\end{IEEEkeywords}

\section{Introduction}
This document is a model and instructions for \LaTeX.
Please observe the conference page limits. 

\section{Background and Related Work}

\subsection{Maintaining the Integrity of the Specifications}

The IEEEtran class file is used to format your paper and style the text. All margins, 
column widths, line spaces, and text fonts are prescribed; please do not 
alter them. You may note peculiarities. For example, the head margin
measures proportionately more than is customary. This measurement 
and others are deliberate, using specifications that anticipate your paper 
as one part of the entire proceedings, and not as an independent document. 
Please do not revise any of the current designations.

\section{Proposed solutions}

\section{Experimental results}

\section{Conclusion}

\begin{thebibliography}{00}
\bibitem{b1} G. Eason, B. Noble, and I. N. Sneddon, ``On certain integrals of Lipschitz-Hankel type involving products of Bessel functions,'' Phil. Trans. Roy. Soc. London, vol. A247, pp. 529--551, April 1955.
\bibitem{b2} J. Clerk Maxwell, A Treatise on Electricity and Magnetism, 3rd ed., vol. 2. Oxford: Clarendon, 1892, pp.68--73.
\bibitem{b3} I. S. Jacobs and C. P. Bean, ``Fine particles, thin films and exchange anisotropy,'' in Magnetism, vol. III, G. T. Rado and H. Suhl, Eds. New York: Academic, 1963, pp. 271--350.
\bibitem{b4} K. Elissa, ``Title of paper if known,'' unpublished.
\bibitem{b5} R. Nicole, ``Title of paper with only first word capitalized,'' J. Name Stand. Abbrev., in press.
\bibitem{b6} Y. Yorozu, M. Hirano, K. Oka, and Y. Tagawa, ``Electron spectroscopy studies on magneto-optical media and plastic substrate interface,'' IEEE Transl. J. Magn. Japan, vol. 2, pp. 740--741, August 1987 [Digests 9th Annual Conf. Magnetics Japan, p. 301, 1982].
\bibitem{b7} M. Young, The Technical Writer's Handbook. Mill Valley, CA: University Science, 1989.
\end{thebibliography}
\vspace{12pt}
\end{document}
